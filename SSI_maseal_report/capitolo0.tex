\chapter{Lo stato dell'arte}
\label{chap:fond}

\begin{minipage}{12cm}\textit{Se lo si desidera, utilizzare questo spazio per inserire un breve riassunto di ci\`o che verr\`a detto in questo capitolo. Inserire solo i punti salienti.}
\end{minipage}

\vspace*{1cm}

\section{Applicazioni simili}
\label{sec:iniziare}

Molto ciriciokhiugiugiu spesso capita che ad un museo o ad una mostra vorremmo
approfondire le informazioni dell'autore o dell'opera che stiamo osservando. Tale motivazione
unita al progresso tecnologico ha portato allo sviluppo di applicazioni che
permettono di far ci\`{o} semplicemente facendo una fotografia.

\subsection{Artfinder}

Artfinder \`{e} un'applicazione per Iphone o Ipad che permette, scattando una
foto ad un quadro, scultura o disegno, di avere tutte le informazioni che si
desiderano. Se la ricerca da esito negativo \`{e} possibile aggiungere
l'opera con i relativi dati. Inoltre ha anche funzioni social:
offre l'oppurtunit\`{a} di condividere le passioni artistiche con amici e
follower.
Artfinder non \`{e} solo un riconoscitore di immagini, ma permette di riconoscere
dove sono in mostra gli artisti preferiti, di avere una preview delle opere
esposte e di creare un'art list. Si possono consultare, inoltre, orari delle
esposizioni e, grazie a un sistema di geolocalizzazione, segnalare tutti i
musei, le gallerie o le mostre che sono facilmente raggiungibili dal luogo in
cui ci si trova.
Se non si � in possesso di uno smartphone o un tablet Apple,
Artfinder \`{e} anche un sito internet sul quale si possono ricercare i quadri, acquistare copie direttamente online oppure creare la tua galleria virtuale.

\subsection{Google Goggles}
 
 Google Goggles \`{e} un'applicazione per il riconoscimento visivo delle
 immagini con tecnologia OCR\footnote{Optical Character Recognition, sono
 programmi dedicati alla conversione di un'immagine contente testo.}. Tale
 applicazione ha molteplici possibilit\`{a} di utilizzo, rilevamente quella di
 ricoscimento delle opere d'arte, la quale \`{e} valsa a Google un accordo con
 il Metropolitan Museum of Art per fornire link diretti al proprio sito sui capolavori in esso esposti.
 Inoltre ha anche le seguenti funzionalit\`{a}:
 \begin{itemize}
   \item trova informazioni scattando una foto su punti di riferimento, libri,
   negozi, quadri etc;
   \item scannerizzando un barcode fornisce dettagli sul prodotto associato ad
   esso;
   \item esegue la scansione dei biglietti da visita e comprende i dati
   pertinenti per crearne un contatto;
   \item risolve i sudoku.
 \end{itemize}
 Tutte le attivit\`{a} riportate sopra si basano sulla possibilit\`{a} di
 riconoscere gli oggetti o il testo nell'immagine catturata dal telefono.
 Esso utilizza tecniche multiple per il riconoscimento delle immagini: in primo
 luogo si cerca di identificare l'oggetto con alcuni algoritmi di riconoscimento
 e lo confronta con delle immagini di un database di Google. Per aiutare
 la ricerca si cerca di trovare del testo nell'immagine utilizzando il
 riconoscimento ottico dei caratteri per avere un'idea migliore di ci\`{o} che
 l'oggetto potrebbe essere. Utilizza anche il GPS per capire dove si trova
 l'utente per filtrare i risultati ricevuti per i luoghi di interesse con
 quelli che sono rilevanti dalla posizione.
Ci sono divesi algoritmi per l'Object Recognition:
\begin{itemize}
	\item \textbf{Edge Detection,} i contorni in un'immagine di solito sono
   robusti al cambiamento di illuminazione/colore. L'esecuzione di
   algoritmi di rilevamento sull'immagine, come Canny Edge
   Detection\footnote{Algoritmo per il riconoscimento dei contorni, utilizza
   un metodo di calcolo multi-stadio per individuare contorni di molti dei tipi
   normalmente presenti nelle immagini reali.}, sono in grado di
   rilevarli nel modello e nell'immagine.
   In seguito vengono confrontati con le possibili soluzioni del modello.
   \item  \textbf{Scale Invariant Feature Transform,} i punti chiave delle
   immagini sono estratti prima da un insieme di immagini di riferimento e poi
   archiviate in un database. Un oggetto \`{e} riconosciuto in una nuova
   immagine comparando individualmente ogni feature di questa con il database
   e cercare candidati che corrispondono alle feature basate sulla distanza euclidea fra i loro vettori di feature.
 \end{itemize}
Google Googles era inizialmente cos\`{i} sensibile che in molti casi poteva
trovare una persona attraverso un'immagine e restituire il link ad un blog che
conduce a lui. Google si rese conto dei problemi di privacy e ora controlla i
risultati in modo da non riconoscere le persone. Essendo ancora in fase di
sperimentazione non tutti i risultati sono accurati, infatti non lavora bene
con mobili e vestiario.
Ci sono comunque diverse accorgimenti per ottenere risultati migliori con Google
Goggles:
\begin{itemize}
  \item scattare foto in ambienti con buona illuminazione;
  \item zoommare ci\`{o} che si vuole fotografare;
  \item utilizzare il pulsante di ritaglio per concentrarsi sull'area
  d'interesse;
  \item usare il telefono con orientamento `paesaggio`;
  \item tenere le mani ferme ed utilizzare sullo schermo il pulsante di scatto.
\end{itemize}



